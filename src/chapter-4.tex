%第四章
\chapter[布局管理]{布局管理}
接下来我们将讲讲控件,组成程序图形界面的“积木块”。怎样布置控件到窗口中?
尽管在\textit{Tkinter}中有三种不同“图形管理器”,但是笔者特别愿意用.grid()图形管理器来布局。这个管理将所有的窗口或框架当做一个表---有行和列的网格。
\begin{itemize}
\item
单元格是一行和一列的交叉区域。
\item
每列最宽的单元格的宽度是该列的列宽。
\item
每行最高的单元格的高度是该行的行高。
\item
控件不可能完全充满单元格的所有空间,你可以对单元格进行指定。你既可以不调整控件外的多余控件,也可以伸展水平方位或垂直方位来使控件填满单元格。
\item
你可以将多个单元格合并生成一个大单元格。
\end{itemize}
当你创建了一个控件,除非你在图形管理器中注册了它否则它将不显示。因此,构造和放置控件的两个步骤进行如下:
\begin{lstlisting}[language=python]
self.thing = tk.Constructor(parent, ...)
self.thing.grid(...)
\end{lstlisting}
\textit{Constructor}是如按钮、框架等控件的类,\textit{Parent}是将被构造的子类控件的父类控件。所有的控件都有.grid()方法,你可以使用它来告诉图形管理器怎么放置控件。

%第4.1节
\section[.grid()方法]{.grid()方法}
将一个控件\textit{W}显示到程序窗口中:
\begin{lstlisting}[language=python]
w.grid(option=value, ...)
\end{lstlisting}
该方法在图形管理器中注册了一个控件\textit{w}---如果不这么做,控件将只存在内部不会显示出来。见表一“.grid()图形管理器参数”:
\begin{longtable}{|l|p{0.85\textwidth}|}

\hline
\textsf
column &
控件放置的列数,从0开始计算。默认值是0。\\ \hline
columnspan & 
通常一个控件只占据网格中的一个单元格。然而,你可以选取行中的多个单元格,并在columnspan选项中设置单元格的数量来整合他们到一个大单元格。例如,\textit{w}.grid(row=0,column=2,columnspan=3),例中将会把\textit{w}控件放置在一个横跨0行2,3,4列的一个单元格中。\\ \hline
in\_ &
注册\textit{w}作为某个控件如\textit{w$_2$}的子类,用法in\_=\textit{w$_2$}。当\textit{w}被创建时,新的\textit{w$_2$}控件必须作为\textit{parent}控件的子类来使用。\\ \hline
ipadx &
内部x padding。这个维度是增加控件内部左边和右边。\\ \hline
ipady &
内部y padding。这个维度是增加控件内部顶部和底部。\\ \hline
padx &
外部x padding。这个维度是增加控件外部左边和右边。\\ \hline
pady &
外部y padding。这个维度是增加控件上部和下部。\\ \hline
row &
你想把控件插入的行数,从0计数。默认是下一个更高的未被占用的行。\\ \hline
rowspan &
通常一个控件只占据网格的一个单元格。你可以选取列内的多个单元格,然后,给选中的单元格设置rowspan选项。本选项和columnspan选项结合使用来抓取一块单元格。例子,\textit{w}.grid(row=3,column=2,rowspan=4,columnspan=5)例中将\textit{w}微件放入一个由3-6列2-6行组成的区域中。\\ \hline
sticky &
本项决定怎样分配单元格中的微件所占空间之外的空间。见下。\\ \hline
\end{longtable}

\begin{itemize}
\item 如果未设置sticky属性,默认会将微件在单元格中居中放置。

\item 你可以使用sticky=NE(右上),SE(右下),SW(左下),或者NW(左上)来布局微件到单元格的四角。

\item 你可以使用sticky=N(中上),E(中右),S(中下),或者W(中左)来布局微件到一边的相对中间。

\item 使用sticky=N+S垂直扩展微件但让它水平居中。

\item 使用sticky=E+W水平扩展微件但让它垂直居中。

\item 使用sticky=N+E+S+W在水平和垂直方位来扩展微件填充单元格。

\item 其它的组合也可使用。例如,sticky=N+S+W将垂直扩展微件并放置微件在相对东(左)边。
\end{itemize}

%第4.2章
\section{其它grid管理方法}
这些grid相关的方法在所有控件上都有定义:
\subsection*{\textsf{\textit{w}.grid\_bbox ( column=None, row=None, col2=None, row2=None )}}
\par{返回一个四元组描述微件w中一些或所有grid系统的边界框。前两个数字返回左上角区域的x和y坐标,后两个数字是宽和高。}
\par{如果你传递行和列变量,返回的限定框描述所在行和列的单元格的区域。如果你也传递了col2和row2参数,返回的限定框描述包含从行column到col2,列row到row2的区域。}
\par{例如,w.grid\_bbox(0,0,1,1)返回四个单元格的限定框,不是一个。}

\subsection*{\textsf{\textit{w}.grid\_forget()}}
\par{本方法使微件w从屏幕上消失。它还存在,只是不可见。你可以使用.grid()使它再次显示,但是它将不会记住它的grid选项。}

\subsection*{\textsf{\textit{w}.grid\_info()}}
\par{返回一个键为w微件选项名字的字典,以及这些选项相应的值。}

\subsection*{\textsf{\textit{w}.grid\_location (x,y)}}
\par{赋予关联的包含的微件一个坐标,本方法返回一个数组(col,row)描述w微件的网格系统的单元格包含的屏幕坐标。}

\subsection*{\textsf{\textit{w}.grid\_propagate()}}
\par{通常,所有微件传送他们的尺寸,意味着他们调节来适应内容。然而,有时你想约束一个微件到确定的尺寸,忽略它内容的尺寸,这样做,调用w.grid\_propagate(0)限制w微件的尺寸。}

\subsection*{\textsf{\textit{w}.grid\_remove()}}
\par{本方法类似.grid\_forget() ,但是它的grid选项会记住 ,所以如果你再.grid()它 ,它将会使用相同的grid配置选项。}

\subsection*{\textsf{\textit{w}.grid\_size()}}
\par{分别在w微件grid系统中返回一个包含行数和列数的二元组。}

\subsection*{\textsf{\textit{w}.grid\_slaves ( row=None, column=None )}}
\par{返回由微件w管理的微件的目录。如果没有提供参数,你将会获得所有被管理的微件的目录。使用row=参数只选择一列微件,或者使用column=参数只选择一行的微件。}

%第4.3章
\section{配置列和行的尺寸}
除非你采取确切的措施,网格列内的微件宽将会等于它的最大宽度,并且网格行内的微}件高将会是最高的单元格的高。控件里的sticky属性只控制控件被放置的地方如果所在单元格未被完全填充满。

\subsection*{\textsf{\textit{w}.columnconfigure ( N, option=value,...)}}
\par{w微件的网格层中, 配置column N以便所给选项有所给的值。对于选项,请看下表。}

\subsection*{\textsf{\textit{w}.rowconfigure ( N, option=value, … )}}
\par{w微件的网格层中,配置row N以便所给的选项有所给的值。对于选项,请看下表。}

以下是用来配置column和row尺寸的选项。\\
\begin{tabular}{|l|p{0.85\textwidth}|}
\hline
minsize&
以像素为最小单位,列和行的最小尺寸。如果列内或行内没有内容,它将不会显示即使你使用了这一选项\\ \hline
pad&
若干像素将会被加入所给的列或行,及以上的列或行中最大的单元格。\\ \hline
weight&
为了使一列或一行可伸展,当重新分配额外的空间时,使用此选项并提供一个值,赋予该列或行相对权重。例如,如果一个微件w包含一个网格层,这些行将会分配3/4的额外空间到第一列及1/4到第二列:
\begin{lstlisting}[language=python]
w.columnconfigure(0, weight=3)
w.columnconfigure(1, weight=1)
\end{lstlisting}
如果没有使用此项,列或行将不会伸展。\\ \hline
\end{tabular}

%第4.4章
\section{让根窗口可重组}
你想让用户能调整你的整个程序窗口大小,并将空出的控件分配给内部的控件吗?只需普通的几个操作就能实现。

这需要用到行和列尺寸管理的方法,在第4.3小节已经提到,“配置列和行大小”(p.7),来使你的\textit{Application}控件的网格可伸缩。然而,那仅仅是不够的。

思考下第二章讨论的小程序,“一个小程序”(p.2),这个程序只包含一个退出按钮。如果你运行此程序,并调整窗口大小,按钮会保持同样的尺寸,并居中于窗口。

以下是小程序内.\_\_createWidgets()方法的替代版本。在这个版本中,退出按钮总是填满可用空间。
\begin{lstlisting}[language=python]
def createWidgets(self):
	top=self.winfo_toplevel()					1
	top.rowconfigure(0,weight=1)	   				2
	top.columnconfigure(0,weight=1)					3
	self.rowconfigure(0,weight=1)					4
	self.columnconfigure(0,weight=1)				5
	self.quit=Button (self, text=”Quit”, command=self.quit )
	self.quit.grid( row=0, column=0, sticky=N+S+E+W)		6
\end{lstlisting}
\begin{enumerate}
\item “top level window''是屏幕上最顶层的窗口。但是,这个窗口不是\textit{Application}的窗口——它是\textit{Application}实例的父级。要获取顶层窗口,在程序中的任何控件上调用.winfo\_toplevel()方法。参看章节,“通用控件方法”(p.97)。
\item 本行代码使top level window的0列网格可伸展。 
\item 本行代码使顶层窗口的0列网格可伸展。
\item 使0行的\textit{Application}控件的网格可伸展。
\item 使0列的\textit{Application}控件的网格可伸展
\item 参数sticky=tk.N+tk.S+tk.E+tk.W使按钮展开填满它所占的单元格
\end{enumerate}
还必须作出一个变化。在构造函数中,如下显示的改变第二行:
\begin{lstlisting}[language=python]
def __init__(self,master=None):
	Frame.__init__(self,master)
	self.grid(sticky=tk.N+tk.S+tk.E+tk.W)
	self.createWidgets()
\end{lstlisting}
sticky=tk.N+tk.S+tk.E+tk.W参数对self.grid()是必要的,因此\textit{Application}控件将会展开填充它所在top-level window网格的单元格。

