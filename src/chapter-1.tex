%第一章
\chapter[Python的一个跨平台用户图形界面]{Python的一个跨平台用户图形界面}
\textit{Tkinter}是Python的一个GUI(graphical user interface)组件。本文档适用于运行在Linux操作系统的X Window系统中的Python2.7和Tkinter8.5。你的版本或稍有差异。
\\
相关参考:
\begin{itemize}

\item Fredrik Lundh, who wrote \textit{Tkinter}, has two versions of his \textit{An Introduction to Tkinter}: \href{http://www.pythonware.com/library/tkinter/introduction/}{a more complete 1999 version}\footnotemark[3] and \href{http://effbot.org/tkinterbook/}{a 2005 version}\footnotemark[4] that presents a few newer features. 

\item \textit{\href{http://www.nmt.edu/tcc/help/pubs/python/}{Python 2.7 quick reference}}\footnotemark[5]{: general information about the Python language}

\item For an example of a sizeable working application (around 1000 lines of code), see huey: \href{http://www.nmt.edu/tcc/help/lang/python/examples/huey/}{\textit{A color and font selection tool}}\footnotemark[6]. 

\end{itemize}

我们将从\textit{Tkinter}可见的部分开始:创建控件并布局在屏幕上。稍后我们将探讨如何将应用程序前端的面板关联到后端逻辑。

\footnotetext[3]{\href{http://www.pythonware.com/library/tkinter/introduction/}{http://www.pythonware.com/library/tkinter/introduction/}}
\footnotetext[4] {\href{http://effbot.org/tkinterbook/}{http://effbot.org/tkinterbook/ }}
\footnotetext[5] {\href{http://www.nmt.edu/tcc/help/pubs/python/ }{http://www.nmt.edu/tcc/help/pubs/python/ }}
\footnotetext[6] {\href{http://www.nmt.edu/tcc/help/lang/python/examples/huey/}{http://www.nmt.edu/tcc/help/lang/python/examples/huey/}}